\documentclass{article}
\author{Semian Detreköy}
\date{\today}
\title{\LaTeX \space \"Ubung}
\begin{document}
\maketitle
\section{Mir f\"allt kein Titel ein}
Und auch kein Text.
\section{Tabelle}
Leistungsnachweis des letzten Jahres:
\begin{table}[h]
\centering
\begin{tabular}{c|c|c|c}
  & Punkte erhalten & Punkte mglich & \% \\
\hline Max & 198 & 250 & 79.2 \\
Moritz & 163 & 250 & 65.2 \\
Semian & 321 & 250 & 128.4 \\
\end{tabular}
\caption{V\"ollig legitime Werte}
\end{table}
\section{Formeln}
\subsection{Pythagoras}
Der Satz des Pythagoras errechnet sich wie folgt:$a^2+b^2=c^2$. Daraus können wir die Länge der Hypothenuse wie folgt berechnen: $c=\sqrt{a^2+b^2}$
\subsection{Summen}
Wir k\"onnen auch die Formel f\"ur eine Summe angeben:
\begin{equation}
s=\sum\limits_{i=1}^{n}i=\frac{n*(n+1)}{2}
\end{equation}
\end{document}
